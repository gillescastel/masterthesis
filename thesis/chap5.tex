\chapter{$h$-cobordism theorem}

TODO: we follow Milnor\sidecite{hcobord}.
Let $f$ be a Morse function on  $(W, V_0, V_1)$ having critical points $p, p'$ of index $\lambda, \lambda+1$ such that  $f(p) < \frac{1}{2 } < f(p')$.
A gradient-like vector field $\xi$ for  $f$ determins in  $V = f^{-1}(\frac{1}{2)}$ a right hand sphere $S_R$ of  $p$ and a left-hand sphere  $S_L'$ of  $p'$.
Note that  $\dim S_R + \dim S_L' = \dim V$.

\begin{theorem}[5.4 First Cancellation theorem]
    TODO recap
\end{theorem}



\begin{definition}[Intersection number]
    Let $M$ and  $M'$ be $r$- and  $s$- dimensional submanifolds of an $(r+s)$-dimensional manifold $V$.
    Suppose $M$ is oriented and  $M'$ is cooriented and their intersection is transverse.
    Let $p \in  M \tcap M'$.
    Then $T_p M$ is oriented both by the orientation of  $M$ and the coorientation of  $M'$.
    If these orientations match, we define the intersection number at $p$ to be $+1$, otherwise we define it to be $-1$.
    The intersection number  $M' \cdot M$ is then defined as the sum of all the intersection numbers at the points in $M \tcap M'$.
\end{definition}
\begin{remark}
    If we change the order, the intersection number might change signs: $M \cdot M' =  \pm M' \cdot M$.
\end{remark}

\begin{marginfigure}
    \centering
    \incfig{intersection-number}
    \caption{
        The intersection number is defined by comparing the orientation of $M$ with the coorientation of $M'$ at the points of transverse intersection. In this case $M' \cdot M = -1 + 1 = 0$.
    }
    \label{fig:intersection-number}
\end{marginfigure}
\begin{prop}
    The intersection number $M \cdot M'$ does not change under deformations of $M$ or an ambient isotopy of  $M'$.
\end{prop}
\begin{proof}
    See TODO
\end{proof}
\begin{eg}
    Consider the intersection of $M  = S^{1}$ and an interval embedded in $\R^2$. Consider the orientation of $M$ and coorientation of $M$ as defined in Figure~\ref{fig:intersection-number}. Then $M' \cdot M  = -1 + 1 = 0$.
    It is also clear that the intersection number does not change under isotopies of the manifolds.
\end{eg}

Now that we have this definition, we are able to state a second stronger cancellation.
It is essentially the same as theorem TODO which allows us to cancel two critical points of consecutive index under certain conditions, including that there is a unique flow line connecting the points.
In the following theorem, we will allow for multiple flow lines connecting the critical points, but we will will assign signs to these trajectories and require that signed count is $\pm 1$.

\begin{theorem}[6.4 Second Cancellation Theorem]
    Let $M$ be a cobordism from  $M_0$ to $M_1$ and $f: M \to  [0,1]$ a Morse function.
    Suppose $f$ has two critical points $p, q$ of index $k$ and $k+1$ with $k \ge  2$ and $k + 1 \le  n-3$.
    Suppose furthermore that $M, M_0, M_1$ are simply connected.
    Lastly, let $\alpha$ be a regular value between $f(p)$ and  $f(q)$, define $V = f^{-1}(\alpha)$ and assume that
    \[
        (\unstable q \cap V) \cdot
        (\stable p \cap V) = \pm 1
    .\] 
    Then we can cancel $p$ and $q$.
    \label{thm:cancel-second}
\end{theorem}
\begin{remark}
    Note that the theorem assumes that $\unstable q$ and $\stable p$ are oriented. 
    This is possible since TODO
\end{remark}
\begin{remark}
    We can also interpret the requirement of having a single flow line (counting signs) as follows.
    Let $M_k = \partial_k$ be the matrix associated to $\partial_k: C_{k} \to  C_{k-1}$ over $\Z$.
    Then the condition is that there is a lonely $\pm 1$ with zeros in the column/row its in.
    The conclusion of the theorem is then that there exists a $f', X'$ such that  $M_k' = \partial_k'$ is the same matrix with this row/column removed.

\end{remark}

\todo{find 2-dimensional example with -1 -1 +1 flowline sign!}
\begin{marginfigure}
    \centering
    \incfig{second-cancellation-theorem-setup}
    \caption{Illustration of symbols used in the statement of Theorem~\ref{thm:cancel-second}. Note the figure is somewhat misleading because of dimensionality reasons.}
    \label{fig:second-cancellation-theorem-setup}
\end{marginfigure}
\begin{remark}
    By changing $f \leadsto -f$, the theorem is also true if we replace the conditions with  $k \ge  3$ and $k + 1 \le  n-2$.
\end{remark}

\begin{remark}
    TODO: The theorem holds even with the single dimension restriction $ n \ge  6$. (Need to check $k = 1$ and $k = n-2$)
\end{remark}

The proof of the second cancellation theorem will reduce the given situation with multiple flow lines connecting $p$ and  $q$ to a situation where there is only one flow line, by repeatedly applying the following theorem by Whitney \sidecite{whitney1944self}
\begin{theorem}[Whitney]
    Let $M$ and  $M'$ be smooth, closed, tranversely intersecting submanifolds of dimensions  $r$ and $s$ in a smooth  $(r+s)$-dimensional manifold $V$.
    Suppose  $M$ is oriented and  $M'$ is cooriented.
    Suppose $r+s \ge 5$, $s\ge 3$ and if $r<3$ suppose that $\pi_1(V - M') \hookrightarrow \pi_1(V)$ is an isomorphism.

    Let $p, q \in  M \tcap M'$ points with opposite intersection number as in Figure~\ref{fig:intersection-points-cancellation} such that there exists a loop $L$ contractible in  $V$ connecting  $p$ smoothly to $q$ in  $M$  and then $q$ smoothly to  $p$ in $M'$ where both arcs miss other intersection points.

    Then there exists an isotopy $h_t$ of the identity $V \to V$ such that
    \begin{enumerate}[(i)]
        \item The isotopy is locally the identity around other intersection points;
        \item At time  $t = 1$,  $M'$ and  $M$ do not intersect in $p$ and $q$  any more.  In other words, $h_1(M) \cap M' = M \cap  M' \setminus \{ p, q\} $.
    \end{enumerate}
\end{theorem}
\begin{marginfigure}
    \centering
    \incfig{intersection-points-cancellation}
    \caption{
    Under certain conditions, we can `cancel' intersection points of opposite intersection number by deforming the manifold $M$ by an isotopy.}
    \label{fig:intersection-points-cancellation}
\end{marginfigure}
\begin{proof}[Proof of the second cancellation theorem]
    Let $X$ be an adopted pseudo-gradient vector field adopted to  $f$ and satisfying the Smale condition.
    Let $S^{u}(q) = \unstable q \cap V$ and $S^{s}(p) = \stable p \cap  V$.

    We now that $S^{u}(q) \cdot S^{s}(p) = \pm 1$, which means that either there is only one flow line connecting $p$ and  $q$, in which case the proof reduces to the first cancellation theorem, or there are multiple flow lines with opposite signs.
    If we can show that the conditions of the theorem of Whitney are satisfied, we can cancel these intersection points pair by pair (by altering $X$) until we have reached the situation of the first cancellation theorem.

    Let $V = f^{-1}(\alpha)$.  If $M$ is simply connected, then $V$ is also simply connected.\sidenote{This follows from applying Seifert--Van Kampen theorem: $\pi_1(V) \cong \pi_1(\stable{p} \cup  V \cup \unstable{q})$ (here we use that $k > 2, n - k \ge 3$).
    Then notice that $\stable{p} \cup V \cup \unstable{q}$ is homotopic to $M$ showing that $\pi_1(V) \cong \pi_1(M) = 1$.}
    If $k \ge 3$, then all the conditions of the theorem are satisfied and we are done.
    If $k = 2$, we need to show that  $\pi_1(V - S^{s}(p)) \cong \pi_1(V)$,
    ore equivalently showing that $\pi_1(V - S^{s}(p))$ is trivial.
    Let $S = \unstable q \cap f^{-1}(0)$.
    Flowing $M_0 \setminus S$ via $X$ gives  $V - S^{s}(p)$, so we need to show that $M_0 \setminus S$ has trivial fundamental group.
    For this we use Seifert--Van Kampen. Let $N$ be a product neighbourhood\sidenote{If the normal bundle of a submanifold is trivial, a tubular neighbourhood is called a product neighbourhood. TODO why does this exists?} of $S$ in $M_0$.
    Note that $S$ is diffeomorphic to $S^{1}$, and $\dim M_0 = n-1$, so the product neighbourhood is diffeomorphic to $ S^1 \times \R^{n-2}$.
    As $k=2$, $n - 2 \ge 4$, so $\R^{n-2} \setminus \{ 0 \} $ has trivial fundamental group. Therefore $N \setminus S \cong S^{1} \times (\R^{n-2}\setminus \{0\})$ has the same fundamental group as $S^{1}$, which is $\Z$.
    This allows us to use Seifert--Van Kampen in the following way:
    $(M_0 \setminus S) \cup N = M_0$, and $\pi_1(M_0) = 1$ $\pi_1(N) = \Z$ $\pi_1(N \setminus S) = \Z$. This implies that $\pi_1(M_0 \setminus S) = 1$, completing the proof.
\end{proof}

Let us finally prove the theorem of Whitney.

\begin{proof}[Proof of Whitney's theorem]
    We will construct the isotopy of $M$ in a local model, constructed as follows.
    \paragraph{Plane model}
    Let $C, C'$ be the arcs in  $M$ and $M'$ connecting $p$ and  $q$ and extend them a little bit either way, as in Figure~\ref{fig:whitneys-theorem-proof-model}.
    \begin{marginfigure}
        \center
        \incfig{whitneys-theorem-proof-model}
        \caption{On the left: the plane model, on the right: the higher dimensional model.}
        \label{fig:whitneys-theorem-proof-model}
    \end{marginfigure}
    For the plane model, let $C_0$ and $ C_1$ be open curves in the plane intersecting transversely in two points, call them $a$ and $b$.
    Let $D$ be the disk with two corners enclosed by $C_0$ and $C_0'$.
    Then there exists an embedding $\phi_1$ of these curves into $M \cup M'$ such that the following holds:
     \begin{itemize}
         \item $\phi_1(C_0) = C$, $\phi_1(C_0') = C'$
         \item $\phi_1(a) = p$, $\phi_1(b) = q$
    \end{itemize}
    \paragraph{Model}
    Now we claim that we can extend this embedding by adding extra dimensions such that the following conditions are satisfied:
    \begin{itemize}
        \item The new embedding $\phi: U \times \R^{r-1} \times \R^{s-1}$ is an extension of $\phi_1|_{U \cap (C_0 \cup C_0')}$.
        \item $\phi^{-1}(M) = (U \cap C_0) \times \R^{r-1} \times 0$
            \item $\phi^{-1}(M') = (U \cap C_0) \times 0\times \R^{s-1}$
    \end{itemize}
    For a detailed proof of this claim, we refer the reader to \sidecite[p.~75]{hcobord}.
    \begin{marginfigure}
        \centering
        \incfig{whitneys-theorem-model-isotopy}
        \caption{The isotopy $G_t$ in the plane model moves $C_0$ below $ C_0'$, i.e. $ G_1(U \cap C_0) \cap C_0' = \O$.}
        \label{fig:whitneys-theorem-model-isotopy}
    \end{marginfigure}
    With this model, the proof of the theorem follows quite rapidly.
    \paragraph{Isotopy in the plane model}
    Let $G_t: U \to U$ be an isotopy in the plane model that when applied to $C_0$, moves it under $C_0'$ as in Figure~\ref{fig:whitneys-theorem-proof-model}.
    More specifically, we require
    \begin{itemize}
        \item $G_0$ is the identity map
        \item $G_t$ is the identity near the boundary of  $U$ for all $t$
        \item  $ G_1(U \cap C_0) \cap C_0' = \O$.
    \end{itemize}
    \paragraph{Isotopy in the model}
    To extend this isotopy to one on $U \times \R^{r-1} \times \R^{s-1}$, define a bump function $\rho: \R^{r-1}\times \R^{s-1} \to  [0,1]$ supported in $\{(x, y)  \mid |x|^2 + |y|^2 \le  1\}$ and set
    \begin{align*}
        H_t: U \times \R^{r-1}\times \R^{s-1} &\longrightarrow  U \times \R^{r-1} \times \R^{s-1}\\
        (u, x, y) &\longmapsto (G_{t \rho(x, y)}, x, y)
    .\end{align*}
    \paragraph{Isotopy of $V$}
    To finally find an isotopy of $V$, define $F_t: V\to V$ such that
    $F_0$ is the identity, $F_t$ is the identity everywhere except away from  $\Im \phi$, and on  $\Im \phi$, define  $F_t = \phi  \circ  H-t  \circ  \phi^{-1}$.
    This finishes the proof.
\end{proof}

Remember how the second cancellation theorem allows us to discard a row/column of the matrix $\partial_k$ under certain conditions.  This raises the question: can we do other row and column operations to this matrix in a geometric way, by which we mean finding a pair $(f', X')$ such that  $M_k'$ is $M_k$ after applying the operation?

\begin{itemize}
    \item Recording rows and columns: just relabelling of critical points
    \item Multiplying a row with $-1$: this works, just change signs
    \item Adding a column to another column ?
        It turns out we can do this geometrically!
\end{itemize}

As starting from a basis we can reach any other basis by doing these row- and column operations, \ldots \sidecite{kosinski2013differential}

\begin{theorem}
    
\end{theorem}

\begin{theorem}[7.8]
    Suppose $(W, V, V')$ is a triad of dimension  $n\ge 6$ with a Morse function with no critical points of indices $0, 1, n-1, n$. Furthermore assume that  $W, V, V'$ are all simply connected (hence orientable) an that  $H_{\bul}(W, V) = 0$.
    Then $(W, V, V')$ is  a product cobordism.
\end{theorem}

\begin{theorem}[8.1]
    \begin{itemize}
        \item If $ H_0(W, V) = 0$, the critical points of index $0$ can be canceled against an equal number of critical points of index $1$
\item Suppose $W$ and  $V$ are simply connected and  $n \ge  5$
    If there are no critical points of index $0$, we can insert for each index  $1$ critical point a pair of auxiliary index  $2$ and  $3$ critical points and cancel the index  $1$ critical points against the auxiliary index  $2$ critical points. (We trade the critical points of index 1 for an equal number of critical points of index 3)
    \end{itemize}
\end{theorem}

\begin{theorem}[9.2 The $h$-cobordism theorem]
    Suppose the triad $(W, V, V')$ has the properties
     \begin{itemize}
        \item $W, V, V'$ are simply connected
        \item  $H_\bul(W, V) = 0$
        \item  $\dim W = n \ge  6$
    \end{itemize}
    Then $W$ is diffeomorphic to  $V \times [0,1]$
\end{theorem}
