\chapter{Applications of Morse homology}
\label{chap:app-morse-homology}


\todo{Now  that we know it's isomorphic, we have some nice theorems: 
    \begin{itemize}
        \item Poincare duality
        \item Mayer-Vietoris
        \item Kunneth
        \item Excision theorem (?)
        \item Hurewicz theorem (?)
        \item Universal coefficient theorem (?)
    \end{itemize}
But it can be nice to see this directly in Morse theory}



\section{Morse inequalities}

\todo{
    There also exists inequalities over $ \Z$, were we have modules instead of vector spaces and need to talk about rank instead of dimension. Maybe add this as well?
    What is derived here is called the weak morse inequalities. In Morse theory Shintaru Fushida-Hardy, strong ones are derived.
    Mention that rank is stil additive functor.
    
    Show that the strong ones are stronger
}

The Morse inequalities state that there is a lower bound on the number of critical points, which only depends on the homology of the manifold:
\begin{theorem}
    Let $f: M \to  \R$ a Morse function. Then
    \[
        \# \Crit f \ge \sum \dim \HMf[k]{M, \Z_2}
    ,\] 
    and more specifically,
    \[
        \# \Crit_k f \ge \dim \HMf[k]{M, \Z_2}
    .\] 
\end{theorem}
\begin{proof}
    This is actually a very straightforward result, following from the fact that \[
    \HMf[k]{M, \Z_2} = \frac{\Ker \partial_k}{\Im \partial_{k-1}},
    \] so $\dim \HMf[k]{M, \Z_2} = \dim \frac{\Ker \partial_k}{\Im \partial_{k-1}} \le  \dim C_k = \# \Crit_k f$.
\end{proof}

A result in similar vein is the following:
\begin{theorem}[Weak Morse inequalities]
    Let $f: M \to  \R$ be a Morse function. Then
    \[
        \sum (-1)^{k} \# \Crit_k f = \sum (-1)^{k} \dim \HMf[k]{M, \Z_2}  = \chi(M)
    .\] 
    Considering this equality modulo $2$, we get
    \[
        \# \Crit f  \equiv \sum \dim \HMf[k]{M, \Z_2} \mod 2
    .\] 
\end{theorem}
\begin{remark}
    This alternating sum is a generalization of the Euler characteristic of a manifold, defined to be
    \[
    \chi(M) = \sum (-1)^{k} \dim H_k(M, \Z_2)
    .\] 

    In the case of polyhedra, this equation reduces to $2 = V - E + F$, where $V$, $E$ and  $F$ are the number of vertices, edges and faces in the polyhedron.
\end{remark}
% \todo{Mention the definition of $\chi(M) = \sum (-1)^{k} H_k$  and moreover that it doesn't depend on the field/ring we are working over!}
\begin{marginfigure}
    \centering
    \incfig{linear-map-rank-nullity-theorem}
    \caption{Visual depiction of the rank-nullity theorem stating that $\dim \Im T + \dim \Ker T = \dim V$.}
    \label{fig:linear-map-rank-nullity-theorem}
\end{marginfigure}

\begin{marginfigure}
    \centering
    \incfig{morse-chain-complex-visualized}
    \caption{
        Top: same illustration as above, this time for the Morse complex with its differential.
        Each orange lines corresponds to a homology group.
        Bottom: The alternating sum of the dimensions of $C_k$ equals the alternating sum of the dimensions of $H_k$.
    }
    \label{fig:morse-chain-complex-visualized}
\end{marginfigure}
\begin{marginfigure}
    \centering
    \incfig{morse-chain-complex-visualized-truncated}
    \caption{Truncating the above picture, we find that the alternating sum of the dimensions of $C_k$ is greater than the alternating sum of the dimensions of  $H_k$. All the thick black lines cancel, exert the one circled, giving rise to the inequality.
    }
    \label{fig:morse-chain-complex-visualized-truncated}
\end{marginfigure}
\begin{proof}
    Use the fact that 
    \[
        \HMf[k]{M, \Z_2} = \frac{\Ker \partial_k}{\Im \partial_{k-1}} \qquad \# \Crit_k f = \dim C_k,
    \]
    together with rank-nullity theorem for vector spaces (we are working over $\Z_2$).

    To represent this proof more visually, have a look at Figure~\ref{fig:linear-map-rank-nullity-theorem}, expressing the rank-nullity theorem, which in this context says that the two slanted lines are parallel, implying that $\dim V - \dim \Ker T = \dim \Im T$.
    Repeating this diagram for $\partial_k$, remembering that  $\partial_k^2 = 0$, gives Figure~\ref{fig:morse-chain-complex-visualized} (top). We have highlighted the dimension of the homology spaces in orange, and we have indicated spaces of the same dimensions with the same type of black thick lines.
    The bottom part of Figure~\ref{fig:morse-chain-complex-visualized} shows that when we consider the alternating some of the dimensions of $C_k$, the thick black lines cancel, leaving us with the alternating sum of the dimensions of $H_k$.

    We can also truncate this argument, considering only a partial alternating sum, illustrated in Figure~\ref{fig:morse-chain-complex-visualized-truncated}.
    In this case, we do not have equality (the reason has been indicated in the figure), but we do have the following result:
\end{proof}
\begin{theorem}[Strong Morse inequalities]
    For any Morse function $f: M \to  \R$ and for any $m = 0, \ldots, n$, the following inequality holds:
    \[
        \sum_{k=0}^{m} (-1)^{k+m} \# \Crit_k f \ge  \sum_{k=0}^{m} (-1)^{k+m} \dim \HMf[k]{M, \Z_2}
    .\] 
\end{theorem}


\todo{Discuss morse inequalities for $\Z$? Torsion rank and rank?}
\todo{Add references to the above sections}

\begin{remark}
    The Morse equalities are strict when $M$ is closed and simply connected $n\ge 6$, meaning that there always exists a Morse function $f: M \to  \R$ such that $\# \Crit_k f = \dim \HMf[k]f$. This is called Smale's theorem \cite[p.~392]{smale2007generalized}.
\end{remark}
\begin{remark}
    There also exists versions of the Morse inequalities when we are working over $\Z$. While $\Z$ is not a field,
    so the dimension of $\Z$-modules is not well defined, it still is a PID, so we can talk about rank and torsion rank.
    %https://encyclopediaofmath.org/wiki/Morse_inequalities
\end{remark}




\section{The Künneth Formula}
The Künneth Formula is a way to relate the homology of a product to the homology of its factors and it states the following:
\begin{prop}[Künneth formula]
    Let $M, N$ be two manifolds. Then
    \[
        \HMf[k]{M \times N} \cong
        \bigoplus_{i+j  = k} \HMf[i]{M} \otimes \HMf[j]{N}
    ,\] 
where homology is taken with coefficients in $ \Z_2$.
\end{prop}
We can also express this in a different way using the Poincaré polynomial.
For this, define $\beta_k(M) = \dim_{\Z_2} \HMf[k]{M;\Z_2}$, the $k$th Betti number and let $P_M(t) = \sum_k \beta_k(M) ^{k}$.
Then the Künneth formula tells us that $P_{M \times N}(t) = P_M(t) P_N(t)$.
\begin{eg}
    We have $P_{S^{1}}(t) = 1 + t$, so $P_{S^{1}\times S^{1}\times S^{1}} = (1 + t)^3 = 1 + 3t + 3t^2 + 1t^3$, exactly the result we found in Example~\ref{eg:homology-of-the-three-torus}.
    More in general, we have that $\beta_k(T^{n})$ is the $k$th coefficient of $(1+t)^{n}$, i.e.\ $\binom{n}{k}$.
\end{eg}
\begin{proof}
    Let $f, g$ and $X, Y$ be two Morse functions, resp.\ pseudo-gradient fields on  $M$ and  $N$.
    Then $f + g$ is a Morse function and $(X, Y)$ is a pseudo-gradient field. If we assume that $X$ and  $Y$ satisfies the Smale condition, then so does $(X,Y)$.
    Critical points of $f+g$ are pairs of critical points of $f$ and $g$ and their indices are sums of the original indices. 
    Furthermore, trajectories of $(X, Y)$ correspond exactly to pairs of trajectories of $X$ and $Y$.
    Now, in order to understand the differential $\partial_{(X, Y)}$ on $M \times N$, we are interested in gradient flow lines that connect critical points $(a,b)$ and  $(c,d)$ whose index differ by exactly one.
    It's clear that the only way this can happen is when $a = c$ or $b = d$.\sidenote{
        If $a\neq c$ and  $b \neq d$, then  $\Ind c \ge  \Ind a + 1$ and $\Ind d \ge  \Ind b + 1$, so $\Ind (c, d) \ge  \Ind (a, b) + 2$.
    }

    When we think about this in terms of directed graphs of critical points, like we did in the example of $T^{3}$, we find that the graph of $M \times N$ is the Cartesian product of the graph of $M$ and the graph of  $N$.
    With these things in mind, it is easy to check that
    \begin{align*}
        \Phi: \bigoplus_{i+j = k} C_i(f) \otimes C_j(g) &\longrightarrow C_k(f+g) \\
        a \otimes b &\longmapsto (a,b)
    \end{align*}
    is an isomorphism of complexes with the following differentials:
    \[
        (C_\bul(f) \otimes C_\bul(g), \partial_X \otimes 1 + 1 \otimes \partial_Y) \xrightarrow{\Phi}   (C_\bul(f+g), \partial_{(X,Y)})
    ,\] 
    where $(C_\bul(f) \otimes C_\bul(g))_k := \bigoplus_{i+j = k} C_i(f) \otimes C_j(g)$.
    Now, taking the homology of both sides, and remembering that the homology of the tensor product complex is the tensor product of the homologies (Proposition~\ref{prop:hom-tensor-is-tensor-hom}), we get exactly what we want:
    \[
        \bigoplus_{i+j = k} \HMf[i]{M; \Z_2} \otimes \HMf[j]{N; \Z_2} \cong \HMf[k]{M \times N; \Z_2}
    .\] 
    Note that we take the homology with coefficients in $\Z_2$ in order to satisfy the conditions of Proposition~\ref{prop:hom-tensor-is-tensor-hom}.
\begin{marginfigure}
    \centering
    \incfig{kunneth-formula}
    \caption{TODO kunneth formula}
    \label{fig:kunneth-formula}
\end{marginfigure}
\end{proof}
