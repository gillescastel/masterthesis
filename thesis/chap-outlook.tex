\chapter*{Outlook}

As the final chapter of this preliminary report, we would like to give an idea of the possible paths to take in the upcoming semester.
With Chapter 2, we are on the verge of defining the Morse complex.
This is an interesting homology theory that is of a very geometric nature.
While everything initially depends on a choice of Morse function $f$ and pseudo-gradient field $X$, we will prove that the homology that arises is independent.
Even more, we will prove that it is isomorphic to the cellular homology.

Being geometric, the Morse complex can be used very intuitively for proving important theorems such as the Poincaré duality (by simply replacing $f$ by $-f$), the K\"unneth formula, etc.
We will also prove the Morse inequalities which tell us that the number of index $k$ critical points of a Morse function is bounded from below with $\beta_k$, the  $k$-th Betti number.

We would also want to expand the section on Heegaard splittings as they are a very useful tool for studying $3$-manifolds. We would like to introduce Heegaard diagrams, lens spaces and prove interesting theorems regarding the characterisation of $3$-manifolds.

Lastly, we will provide the proof of cancellation of two critical points in much more detail. Especially if we would focus on the $h$-cobordism theorem and higher dimensional Poincaré conjecture, this would be relevant.
Another option is to focus on Floer homology and the Arnold conjecture, following the second part of the book of Audin and Damian.
